\begin{abstract}
    This work analyzes the student transportation network of Izmir Institute of Technology (IZTECH) using graph theory and algorithms. The primary objective is to minimize the total transportation costs and the average length of the service routes while considering the service capacity and privacy of the students' personal address information. The determination of the necessary number of services is another research interest. To address these research goals, we first create a synthetic data set of 2000 student addresses that are distributed around the most favoured residential areas in Izmir.
    Further, the synthetic data set of student addresses transferred to a graph space by using complete graph, Delaunay Triangulation, Gabriel graph and K-nearest neighbor graph construction techniques. To determine service routes, spectral clustering, the Leiden Algorithm and Multi view Anchor Graph-based Clustering(MVAGC) are performed and the shortest paths are evaluated for every cluster associated with every service. Moreover, the isolated vertices of the distant addresses, that noticeably increase the length of route, are detected using a K Nearest Neighbor distance based outlier detection method. Experimental evaluations analyzing the performance of different graph construction and clustering method pairs demonstrate the importance of the graph construction where sparse graphical models are computationally advantageous tools. Among variety of graph construction and clustering pairs, Gabriel graph construction and Leiden algorithm shows the best transportation cost performance with daily 115 095 TL which even reduced to daily 110 364 TL by performing outlier detection. Compared to performing the Leiden algorithm on the complete graph construction, incorporating sparse representation and outlier detection reduced the total cost approximately 11.0\%. Experimental results comprising different graph construction, clustering and outlier detection pairs demonstrate that graph-based analysis of the transportation network of IZTECH is a promising approach as it naturally determines the necessary number of services, provides robustness to the isolated addresses, minimizes the transportation cost and the route lengths. Beyond these advantages, the graph-based analysis provides adaptiveness to different address information and preserve privacy by using address information without students' personal information.
    \end{abstract}
    
    
    
    
    
    
    
    
    
    
    
    
    
    
    
    
    
    
    
    
    
    
    
    
    
    
    
    
    