\chapter{Introduction}
\label{ch:introduction} % label the chapters (and sections) such that you can refer to them

Efficient transportation is a critical component of university campus life, impacting student accessibility, operational costs, and environmental sustainability. For institutions like the Izmir Institute of Technology (IZTECH), serving a large student body spread across a major metropolitan area like Izmir presents significant logistical challenges. These include the inherent complexity of optimizing routes for a dispersed population with varying demand, the need for computationally tractable solutions for large-scale networks, and the impact of real-world data imperfections such as outliers. Optimizing the student transportation network is essential not only for managing costs but also for improving the overall student experience. This thesis addresses these challenges by systematically applying and evaluating graph theory and advanced clustering algorithms to design an optimized bus routing system for IZTECH students.

The primary goal of this research is to develop and evaluate a methodology for minimizing total transportation costs while adhering to practical constraints, such as varying bus capacities (10-50 students per vehicle). We utilize a synthetically generated dataset representing the geographical distribution of approximately 2000 IZTECH students across Izmir. This dataset forms the basis for constructing various graph representations of the transportation network, allowing for a controlled investigation of algorithmic performance.

Addressing the need for efficient network representation, this study systematically investigates the effectiveness of different graph construction techniques. We move from a dense complete graph representation, which can be computationally prohibitive, to sparser, more tractable models like Delaunay Triangulation, Gabriel Graphs, and K-Nearest Neighbors (KNN) graphs. The impact of graph sparsity on the feasibility, computational efficiency, and quality of routing solutions is a key area of exploration, as appropriate sparsity can reveal underlying network structures crucial for optimization.

Furthermore, to effectively group students into viable bus routes, we evaluate the performance of several state-of-the-art graph clustering algorithms – namely Spectral Clustering, the Leiden Algorithm, and Multi-view Anchor Graph-based Clustering (MVAGC). These algorithms are chosen for their diverse approaches to community detection, offering a comprehensive analysis of their suitability for this transportation routing problem. The interplay between graph representation and clustering algorithm performance is analyzed in detail. Recognizing the potential impact of anomalous data on network efficiency, we also incorporate and assess a K-Nearest Neighbor (KNN) distance-based outlier detection method as a preprocessing step, aiming to enhance the robustness and real-world applicability of the routing solutions.

The main contributions of this thesis lie in the systematic application and comparative analysis of various graph construction and clustering techniques for the specific problem of university student transportation optimization, the evaluation of sparsity's role in this context, and the assessment of an outlier detection strategy to improve route efficiency. This introductory chapter outlines the scope and objectives of the thesis. Chapter~\ref{ch:basics} provides the necessary theoretical background on graph theory, sparsity, clustering algorithms, and shortest path computation. Chapter~\ref{ch:method} details the specific methodologies employed, including data generation, graph construction, clustering algorithm implementation, and the outlier detection process. Chapter~\ref{ch:experiments} presents the comprehensive experimental evaluation, comparing the performance of different method combinations based on metrics like total cost, number of routes, and computational time. Finally, Chapter~\ref{ch:conclusions} summarizes the key findings, discusses the methodological contributions, and suggests avenues for future research in transportation network optimization.




