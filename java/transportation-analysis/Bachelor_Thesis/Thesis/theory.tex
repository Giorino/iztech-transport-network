\chapter{Theory \& Background}
\label{ch:basics}
This chapter introduces the fundamental concepts and theories used in transportation network analysis. First, we explore various graph construction methods used to represent transportation networks. Then, we examine different clustering algorithms applied to identify communities within these networks. Finally, we discuss the mathematical foundations of these clustering algorithms, with a particular focus on eigenvectors and spectral methods.

\section{Graph Theory and Transportation Networks}
\label{se:GraphTheory}
Transportation networks are naturally represented as graphs, where nodes typically represent locations or intersections, and edges represent the connections between them, such as roads, paths, or routes. Graph theory provides a powerful mathematical framework for analyzing and understanding the structure, patterns, and properties of these networks.

\subsection{Graph Representation and Terminology}

A graph $G$ is formally defined as an ordered pair $G = (V, E)$ comprising a set $V$ of vertices or nodes and a set $E$ of edges, which are 2-element subsets of $V$. In the context of transportation networks:

\begin{itemize}
    \item \textbf{Vertices (Nodes):} Represent geographical locations such as intersections, bus stops, or points of interest.
    \item \textbf{Edges:} Represent the connections between locations, such as roads, railway lines, or paths.
    \item \textbf{Weight:} Edges can be weighted to represent distances, travel times, costs, or other metrics.
    \item \textbf{Direction:} Edges can be directed (one-way streets) or undirected (two-way streets).
\end{itemize}

The adjacency matrix $A$ of a graph is a square matrix where each element $A_{ij}$ represents the connection between vertices $i$ and $j$:

\begin{equation}
    A_{ij} = 
    \begin{cases} 
        w_{ij} & \text{if there is an edge from vertex $i$ to vertex $j$ with weight $w_{ij}$} \\
        0 & \text{otherwise}
    \end{cases}
\end{equation}

For an unweighted graph, $w_{ij} = 1$ for all edges.

\subsection{Graph Construction Methods}
\label{subsec:GraphConstructionMethods}

Multiple approaches exist for constructing graphs to represent transportation networks. The choice of method depends on the specific requirements and constraints of the analysis. Here, we discuss several common methods:

\subsubsection{Complete Graph}
A complete graph connects every pair of vertices with an edge. In a transportation context, this would represent a fully connected network where direct travel is possible between any two locations. While this approach is simple, it does not accurately represent real-world constraints and can lead to computationally intensive analyses for large networks.

\begin{equation}
    E = \{(u, v) \mid u, v \in V, u \neq v\}
\end{equation}

\subsubsection{K-Nearest Neighbors Graph}
In this approach, each vertex is connected to its $k$ nearest neighbors according to some distance metric (typically Euclidean distance for geographic networks). This method creates a sparse graph where each location is connected only to its closest locations, which often better represents the local connectivity patterns of transportation networks.

\begin{equation}
    E = \{(u, v) \mid v \in \text{kNN}(u) \text{ or } u \in \text{kNN}(v)\}
\end{equation}

where $\text{kNN}(u)$ represents the $k$ nearest neighbors of vertex $u$.

\subsubsection{Delaunay Triangulation}
Delaunay triangulation creates a graph by connecting vertices such that no vertex lies inside the circumcircle of any triangle formed by three connected vertices. This method preserves local connectivity while avoiding crossing edges, making it useful for geographic applications.

\subsubsection{Gabriel Graph}
The Gabriel Graph is a subgraph of the Delaunay triangulation. An edge connects two vertices $u$ and $v$ if and only if the circle with diameter $uv$ contains no other vertices. The Gabriel Graph tends to preserve important connections while reducing the number of edges compared to a complete graph.

\begin{equation}
    E = \{(u, v) \mid d^2(u, v) < d^2(u, w) + d^2(v, w) \text{ for all } w \in V, w \neq u, w \neq v\}
\end{equation}

where $d(u, v)$ is the distance between vertices $u$ and $v$.

\subsubsection{Road Network Extraction}
For many transportation applications, the graph is constructed directly from existing road network data, such as OpenStreetMap (OSM). This approach creates a graph that accurately represents the actual road infrastructure, including one-way streets, turn restrictions, and different road types.

\section{Clustering Methods in Transportation Networks}
\label{se:ClusteringMethods}

Clustering in transportation networks involves partitioning the network into cohesive groups or communities based on connectivity patterns, geographical proximity, or other relevant factors. These clusters can represent neighborhoods, functional regions, or service areas within a transportation system.

\subsection{Community Detection and Clustering}

Community detection algorithms identify groups of nodes that are more densely connected internally than with the rest of the network. In transportation networks, these communities often correspond to natural geographic or functional regions. The quality of a clustering can be measured using metrics like modularity, which quantifies the density of connections within communities compared to connections between communities:

\begin{equation}
    Q = \frac{1}{2m} \sum_{i,j} \left[A_{ij} - \frac{k_i k_j}{2m}\right] \delta(c_i, c_j)
\end{equation}

where $A_{ij}$ is the edge weight between vertices $i$ and $j$, $k_i$ is the sum of weights of edges attached to vertex $i$, $m$ is the sum of all edge weights, $c_i$ is the community to which vertex $i$ belongs, and $\delta(c_i, c_j)$ is 1 if $c_i = c_j$ and 0 otherwise.

\subsection{Spectral Clustering}
\label{subsec:SpectralClustering}

Spectral clustering uses the eigenvalues and eigenvectors of matrices derived from the graph (such as the Laplacian matrix) to perform dimensionality reduction before clustering. This approach is particularly effective for finding natural clusters in complex networks.

The Laplacian matrix $L$ of a graph is defined as $L = D - A$, where $A$ is the adjacency matrix and $D$ is the degree matrix (a diagonal matrix where $D_{ii}$ is the degree of vertex $i$). The normalized Laplacian is defined as:

\begin{equation}
    L_{\text{norm}} = D^{-1/2} L D^{-1/2} = I - D^{-1/2} A D^{-1/2}
\end{equation}

The eigenvectors of the Laplacian matrix corresponding to the smallest non-zero eigenvalues provide a lower-dimensional embedding of the graph that preserves its clustering structure. This embedding is then clustered using a standard algorithm like $k$-means.

\subsection{Community Detection in Transportation Networks}
\label{subsec:CommunityDetection}

Several algorithms have been specifically adapted or developed for community detection in transportation networks:

\subsubsection{Leiden Algorithm}
The Leiden algorithm improves upon the Louvain algorithm for community detection by ensuring well-connected communities. It uses a fast local move approach where nodes are moved to neighboring communities if this improves modularity, followed by a refinement phase to ensure well-connected communities.

\subsubsection{Multi-view Anchor Graph-based Clustering (MVAGC)}
MVAGC constructs multiple views of the graph using different features or distance metrics and then integrates these views through an anchor graph representation. This approach can effectively capture both geographic proximity and transportation connectivity patterns.

\section{Mathematical Foundations of Clustering Algorithms}
\label{se:MathematicalFoundations}

The effectiveness of clustering algorithms often relies on fundamental mathematical concepts, particularly in linear algebra and spectral graph theory.

\subsection{Eigenvectors and Eigenvalues in Graph Analysis}
\label{subsec:Eigenvectors}

Eigenvectors and eigenvalues play a crucial role in spectral clustering and community detection. For a square matrix $M$, an eigenvector $v$ and its corresponding eigenvalue $\lambda$ satisfy:

\begin{equation}
    M v = \lambda v
\end{equation}

In graph analysis, the eigenvectors of the adjacency matrix or Laplacian matrix capture important structural properties of the graph:

\begin{itemize}
    \item The eigenvector corresponding to the largest eigenvalue of the adjacency matrix (the principal eigenvector) captures the centrality of nodes.
    \item The eigenvectors corresponding to the smallest non-zero eigenvalues of the Laplacian matrix capture the community structure of the graph.
\end{itemize}

\subsection{Spectral Graph Theory and the Laplacian Matrix}
\label{subsec:SpectralGraphTheory}

Spectral graph theory studies the properties of a graph through the eigenvalues and eigenvectors of its associated matrices, primarily the Laplacian matrix.

The multiplicity of the eigenvalue 0 of the Laplacian matrix equals the number of connected components in the graph. The second smallest eigenvalue measures how well-connected the graph is, with larger values indicating better connectivity.

