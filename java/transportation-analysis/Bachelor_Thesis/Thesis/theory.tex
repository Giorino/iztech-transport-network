\chapter{Theoretical Background}
\label{ch:basics}
This chapter introduces the fundamental concepts and theories used in transportation network analysis. First, we explore basic graph theory concepts essential for understanding networks. Then, we examine how transportation systems can be represented as graphs. Finally, we discuss clustering algorithms applied to identify communities within these networks, with focus on their applications in transportation systems .

\section{Fundamental Concepts in Graph Theory}
\label{se:FundamentalConcepts}

\subsection{Basic Definitions}
\label{subsec:BasicDefinitions}

A graph $G$ is formally defined as an ordered pair $G = (V, E)$ comprising a set $V$ of vertices or nodes and a set $E$ of edges, which are 2-element subsets of $V$ . The fundamental components of a graph include:

\begin{itemize}
    \item \textbf{Vertices (Nodes):} Represent distinct entities in the network.
    \item \textbf{Edges:} Represent the connections or relationships between vertices.
    \item \textbf{Weight:} Edges can be weighted to represent distances, travel times, costs, or other metrics.
    \item \textbf{Direction:} Edges can be directed (one-way) or undirected (two-way).
\end{itemize}

Graphs can be classified as directed or undirected, weighted or unweighted, simple or multi-graphs, depending on their properties .

\subsection{Affinity Matrix Construction}
\label{subsec:AffinityMatrix}

The adjacency matrix (also known as the affinity matrix) $\mat{A}$ of a graph is a square matrix where each element $\mat{A}_{ij}$ represents the connection between vertices $i$ and $j$ :

\begin{equation}
    \mat{A}_{ij} = 
    \begin{cases} 
        w_{ij} & \text{if there is an edge from vertex $i$ to vertex $j$ with weight $w_{ij}$} \\
        0 & \text{otherwise}
    \end{cases}
\end{equation}

For an unweighted graph, $w_{ij} = 1$ for all edges. The adjacency matrix provides a complete mathematical representation of the graph and serves as the foundation for many graph algorithms and analyses .

Other important matrices derived from the adjacency matrix include the edge weight matrix $\mat{D}$, a matrix where each entry $\mat{D}_{ij}$ represents the weight of the edge connecting vertex $i$ to vertex $j$ (with $\mat{D}_{ij} = 0$ if no edge exists), the Laplacian matrix $\mat{L} = \mat{D} - \mat{A}$, which captures many structural properties of the graph, and the normalized Laplacian $\mat{L}_{\text{norm}} = \mat{D}^{-1/2}\mat{L}\mat{D}^{-1/2} = \mat{I} - \mat{D}^{-1/2}\mat{A}\mat{D}^{-1/2}$, which is commonly used in spectral clustering.

\section{Transportation Networks}
\label{se:TransportationNetworks}

Transportation networks are naturally represented as graphs, where nodes typically represent locations or intersections, and edges represent the connections between them, such as roads, paths, or routes . In transportation context:

\begin{itemize}
    \item \textbf{Vertices:} Represent geographical locations such as intersections, bus stops, or points of interest.
    \item \textbf{Edges:} Represent the connections between locations, such as roads, railway lines, or paths.
\end{itemize}

\subsection{Graph Construction Methods for Transportation Networks}
\label{subsec:GraphConstructionMethods}

Multiple approaches exist for constructing graphs to represent transportation networks . The choice of method depends on the specific requirements and constraints of the analysis:

\subsubsection{Complete Graph}
A complete graph connects every pair of vertices with an edge . In a transportation context, this would represent a fully connected network where direct travel is possible between any two locations.

\begin{equation}
    E = \{(u, v) \mid u, v \in V, u \neq v\}
\end{equation}

\subsubsection{K-Nearest Neighbors Graph}
In this approach, each vertex is connected to its $k$ nearest neighbors according to some distance metric . This method creates a sparse graph where each location is connected only to its closest locations.

\begin{equation}
    E = \{(u, v) \mid v \in \text{kNN}(u) \text{ or } u \in \text{kNN}(v)\}
\end{equation}
where $\text{kNN}(u)$ represents the $k$ nearest neighbors of vertex $u$.

\subsubsection{Delaunay Triangulation}
Delaunay triangulation creates a graph by connecting vertices such that no vertex lies inside the circumcircle of any triangle formed by three connected vertices . This method preserves local connectivity while avoiding crossing edges, making it useful for geographic applications.

\subsubsection{Gabriel Graph}
The Gabriel Graph is a subgraph of the Delaunay triangulation . An edge connects two vertices $u$ and $v$ if and only if the circle with diameter $uv$ contains no other vertices.

\begin{equation}
    E = \{(u, v) \mid d^2(u, v) < d^2(u, w) + d^2(v, w) \text{ for all } w \in V, w \neq u, w \neq v\}
\end{equation}
where $d(u, v)$ is the distance between vertices $u$ and $v$.

\subsubsection{Road Network Extraction}
For many transportation applications, the graph is constructed directly from existing road network data, such as OpenStreetMap (OSM) . This approach creates a graph that accurately represents the actual road infrastructure.

\section{Clustering Transportation Networks}
\label{se:ClusteringMethods}

Clustering in transportation networks involves partitioning the network into cohesive groups or communities based on connectivity patterns, geographical proximity, or other relevant factors . These clusters can represent neighborhoods, functional regions, or service areas within a transportation system.

\subsection{Spectral Clustering}
\label{subsec:SpectralClustering}

Spectral clustering uses the eigenvalues and eigenvectors of matrices derived from the graph to perform dimensionality reduction before clustering . This approach is particularly effective for finding natural clusters in complex networks. The algorithm is summarized in Algorithm~\ref{alg:spectral_clustering}.

\begin{algorithm}[H]
\caption{Spectral Clustering}
\label{alg:spectral_clustering}
\begin{algorithmic}[1]
\Require Graph $G = (V, E)$, number of clusters $k$
\Ensure Cluster assignments for vertices $V$

\State Construct the Adjacency Matrix $\mat{A}$ for the graph $G$.
\State Construct the Degree Matrix $\mat{D}$ where $\mat{D}_{ii} = \sum_j \mat{A}_{ij}$.
\State Calculate the Laplacian Matrix $\mat{L} = \mat{D} - \mat{A}$.
\State Compute the Normalized Laplacian $\mat{L}_{\text{norm}} = \mat{D}^{-1/2} \mat{L} \mat{D}^{-1/2} = \mat{I} - \mat{D}^{-1/2} \mat{A} \mat{D}^{-1/2}$.
\State Find the $k$ eigenvectors $\vect{u}_1, \vect{u}_2, \dots, \vect{u}_k$ corresponding to the $k$ smallest non-zero eigenvalues of $\mat{L}_{\text{norm}}$.
\State Form the matrix $U \in \R^{|V| \times k}$ with the eigenvectors $\vect{u}_1, \dots, \vect{u}_k$ as columns.
\State Let $\vect{y}_i \in \R^k$ be the vector corresponding to the $i$-th row of $U$.
\State Cluster the points $(\vect{y}_i)_{i=1, \dots, |V|}$ into $k$ clusters $C_1, \dots, C_k$ using the $k$-means algorithm.
\State Assign vertex $v_i$ to cluster $C_j$ if row $\vect{y}_i$ was assigned to cluster $C_j$.

\end{algorithmic}
\end{algorithm}

The eigenvectors of the Laplacian matrix capture important structural properties of the graph:
\begin{itemize}
    \item The multiplicity of the eigenvalue 0 equals the number of connected components in the graph
    \item The second smallest eigenvalue (Fiedler value) measures how well-connected the graph is
    \item The eigenvectors corresponding to the smallest non-zero eigenvalues reveal the community structure
\end{itemize}

\subsection{Leiden Algorithm}
\label{subsec:LeidenAlgorithm}

The Leiden algorithm improves upon the Louvain algorithm for community detection by ensuring well-connected communities and optimizing modularity (Eq.~\eqref{eq:modularity}) . The process is outlined in Algorithm~\ref{alg:leiden}.

\begin{algorithm}[H]
\caption{Leiden Algorithm}
\label{alg:leiden}
\begin{algorithmic}[1]
\Require Graph $G = (V, E)$, initial partition $P$ (optional)
\Ensure Final partition $P_{\text{final}}$ maximizing modularity $Q$

\State Initialize partition $P$ (e.g., each node in its own community).
\State Set \texttt{converged} = \texttt{false}.
\While{not \texttt{converged}}
    \State \textbf{Local Moving Phase}
    \State Set \texttt{moved\_nodes} = \texttt{true}.
    \While{\texttt{moved\_nodes}}
        \State Set \texttt{moved\_nodes} = \texttt{false}.
        \ForAll{node $i \in V$}
            \State Find neighboring community $C_j$ that maximizes $\Delta Q$ (Eq.~\eqref{eq:modularitygain}) for moving $i$ to $C_j$.
            \If{max $\Delta Q > 0$}
                \State Move node $i$ to community $C_j$.
                \State Update partition $P$.
                \State Set \texttt{moved\_nodes} = \texttt{true}.
            \EndIf
        \EndFor
    \EndWhile
    \State \textbf{Refinement Phase}
    \State Create refined partition $P'$ based on $P$.
    \ForAll{community $C \in P$}
        \State Partition $C$ into subcommunities locally (ensures well-connected).
        \State Add refined subcommunities of $C$ to $P'$.
    \EndFor
    \State Update $P = P'$.
    \State \textbf{Aggregation Phase}
    \If{no change in partition $P$ compared to previous iteration}
        \State Set \texttt{converged} = \texttt{true}.
    \Else
        \State Create aggregated graph $G'$ where each node represents a community in $P$.
        \State Set edge weights in $G'$ based on inter-community edge weights in $G$.
        \State Set $G = G'$ for the next iteration.
        \State Update node mappings to reflect aggregation.
    \EndIf
\EndWhile
\State Set $P_{\text{final}} = P$.
\end{algorithmic}
\end{algorithm}

The key innovations of the Leiden algorithm over its predecessors include:
\begin{itemize}
    \item Faster convergence due to more efficient community detection
    \item Guaranteed well-connected communities through the refinement phase
    \item Ability to avoid getting trapped in poor local optima by allowing more flexible node movement
    \item Proven asymptotic guarantees for identifying optimal partitions
\end{itemize}

These properties make Leiden particularly suitable for analyzing transportation networks, where identifying meaningful, well-connected communities is essential for understanding regional connectivity patterns and service areas .

\subsection{Multi-view Anchor Graph-based Clustering (MVAGC)}
\label{subsec:MVAGC}

MVAGC constructs multiple views of the graph using different features or distance metrics and then integrates these views through an anchor graph representation . This approach can effectively capture both geographic proximity and transportation connectivity patterns.

The algorithm employs anchor points to approximate the full graph, reducing computational complexity while preserving the essential structure of the data . This makes it particularly suitable for large-scale transportation networks where computational efficiency is important.

